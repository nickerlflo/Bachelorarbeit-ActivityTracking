\chapter{Implementierung}
\label{cha:implementierung}
In diesem Kapitel werden Teile der im Kapitel \ref{cha:konzept} beschriebenen Komponenten umgesetzt. Dabei werden exemplarisch Beispiele für ausgewählte Komponenten vorgestellt.  
Die vollständige Implementierung ist deutlich komplexer, die zugrunde liegenden Konzepte bleiben jedoch identisch, weshalb eine exemplarische Darstellung ausreichend ist.

\section{Konfiguration}
\label{sec:configuration_impl}

\subsection{Ermitteln der Provider}
% Assemblies Referenzieren
% Assembly Resolver Refernzieren
% Reflection Referenzieren
% Nullreferenz Types Referenzieren

\subsection{Fluent API mit Builder}
%Beispiel Settings
%Beispiel Builder
%Beispiel Fluent API in Anwendung erwähnen

\section{Daten- und Aktionsermittlung}
\label{sec:data_collection_impl}

\subsection{Aktion Provider Template}

\subsection{Auftragsabarbeitung}

\section{Filterung und Extraktion}
\label{sec:data_extraction_impl}

%Asynchrone Programmierung Referenzieren

\subsection{Verarbeitungskette Ablauf}

\subsection{Extraktion Beispiel}

\section{Integration WPF}
\label{sec:integration_wpf_impl}

\subsection{Schritt zur Integration}
%MVVM Refernzieren
%Erweiterungen Referenzieren

\subsection{Beispiel Aktionen Provider}
%Eventsystem Referenzieren

\section{Integration Windows Forms}
\label{sec:integration_winforms_impl}

\subsection{Schritt zur Integration}
%Aufgabendelegation Referenzieren

\subsection{Beispiel Daten Provider}
%Referenz auf Reflection

