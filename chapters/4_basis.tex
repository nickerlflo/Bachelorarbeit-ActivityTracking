\chapter{Bestehendes System}
\label{cha:basis}
Da die Arbeit bereits in ein bestehendes System integriert wird, gibt es Möglichkeiten, bereits vorhandenen Code wiederzuverwenden. Um die Implementierung (Kapitel \ref{cha:implementierung}) nachvollziehen zu können, werden in diesem Kapitel einzelne bestehende Konzepte vorgestellt, die dabei helfen, das Framework zu integrieren und weiterzuentwickeln.

\section{MVVM spezifische Erweiterungen}
\label{sec:mvvm_extensions}
Grundsätzlich basiert das WPF-System (siehe Unterabschnitt \ref{subsec:WPF}) auf dem in Unterabschnitt \ref{fig:mvvm_pattern} beschriebenen MVVM-Pattern. Würde jedoch die gesamte Logik direkt in die Vererbungshierarchie der ViewModels integriert werden, würde dies die Flexibilität erheblich einschränken und die Komplexität deutlich erhöhen. Daher wurde das MVVM-Pattern an die spezifischen Anforderungen der Anwendung angepasst.

\subsection{Presenter}
Bindings ermöglichen zwar einen relativ einfachen Datenaustausch zwischen View und ViewModel, jedoch treten häufig Situationen auf, in denen Daten das Verhalten der View beeinflussen oder Aktionen im ViewModel angestoßen werden müssen, die sich über einfache Commands nicht mehr abbilden lassen. In der betreffenden WPF-Anwendung kommen daher Presenter zum Einsatz, die die Kommunikation zwischen ViewModel und View sowohl auf Basis von Daten als auch über Events ermöglichen.


\subsection{ViewModel Extensions}
\label{subsec:viewmodel_extensions}

\subsection{Presenter und Presenter Extensions}
\label{subsec:presenter_extensions}

\section{MVP spezifische Erweiterungen}
\label{sec:mvp_extensions}

\subsection{Embedded Presenter}
\label{subsec:embedded_presenter}