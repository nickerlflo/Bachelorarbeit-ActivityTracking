%%% Dateikodierung: UTF-8

%%% Magic Comments zum Setzen der korrekten Parameter in kompatiblen IDEs
% !TeX encoding = utf8
% !TeX program = pdflatex 
% !TeX spellcheck = de_DE
% !BIB program = biber

\RequirePackage[utf8]{inputenc} % bei Verw. von lualatex oder xelatex entfernen!
\RequirePackage{hgbpdfa}				% Remove if NO PDF/A output is desired

\documentclass[type=bachelor,theme=default,language=german,titlelanguage=german,smartquotes]{hgbthesis}
% Zulässige Optionen in [..]: 
%    Typ der Arbeit (type=): 'master' (default), 'bachelor', 'diploma', 'phd', 'internship'
%    Theme der Titelseite (theme=): 'default' (default), 'fhooe24'
%    Als Exposé verwenden: 'proposal' oder 'proposal=true' 
%    Hauptsprache im Dokument (language=): 'german' (default), 'english'
%    Sprache der Titelseite (titlelanguage=): 'german', 'english' (default is main language)
%    Umwandlung in typografische Anführungszeichen: 'smartquotes'
%    APA Zitierstil: 'apa'
%    Layout: 'oneside' (einseitig, default), 'twoside' (zweiseitig)
%%%-----------------------------------------------------------------------------

\graphicspath{{images/}}  % Verzeichnis mit Bildern und Grafiken
\bibliography{bib/references} % Biblatex-Literaturdatei (hgbreferences.bib)

%%%-----------------------------------------------------------------------------
\begin{document}
%%%-----------------------------------------------------------------------------

%%%-----------------------------------------------------------------------------
% Angaben für die Titelei (Titelseite, Erklärung etc.)
%%%-----------------------------------------------------------------------------

\title{Aktivitäts-Tracking-Framework}
\subtitle{Design und Implementierung}
\author{Florian Wagner}

\programtype{Fachhochschul-Bachelorstudiengang} % oder Fachhochschul-Bachelorstudiengang
\programname{Software Engineering}
\institution{Fachhochschule Oberösterreich}

\placeofstudy{Hagenberg}
\dateofsubmission{2026}{07}{01} % {JJJJ}{MM}{TT}

% Liste der Betreuungspersonen, bis zu 4 sind möglich, Titel in [] ist optional
\advisor{FH-Prof.~Dipl.-Ing.~(FH)~Dr.~Josef~Pichler}
%\advisor[Zweitbetreuerin]{FH-Prof.\textsuperscript{in} Susanna A.~D. Visor, PhD}

%\license{cc}      % Unter Creative Commons Lizenz veröffentlichen (empfohlen)
\license{strict} % Restriktieve Lizenz, "Alle Rechte vorbehalten"

%%%-----------------------------------------------------------------------------
\frontmatter                                       % Titelei (röm. Seitenzahlen)
%%%-----------------------------------------------------------------------------

\maketitle
\tableofcontents

\chapter{Vorwort}
In die vorliegende Arbeit habe ich viel Energie und Leidenschaft investiert, stets mit dem Ziel, dass dieses Softwaresystem praxistauglich wird und einen wertvollen Beitrag für meinen Arbeitgeber und unsere Kunden leistet. Gleichzeitig war es mir ein Anliegen, mich persönlich weiterzuentwickeln und aus diesem Projekt zu lernen.\\
\\
Ich möchte mich daher für die interessante Aufgabenstellung und die gute Zusammenarbeit bei meinem Arbeitgeber RZL Software GmbH bedanken. Besonders möchte ich meine Wertschätzung dem Team aussprechen, das mich auf diesem Weg unterstützt hat.\\
\\
Ein besonderer Dank gilt meinem Betreuer FH-Prof. Dipl.-Ing. (FH) Dr. Josef Pichler für die ausgezeichnete Betreuung, seine Zeit und die hilfreichen Ratschläge. Aus dieser Zusammenarbeit konnte ich viel lernen und wertvolle Erfahrungen sammeln.

 % Ein Vorwort ist optional
\chapter{Kurzfassung}
\label{cha:abstract}

\begin{german}

Der schnelle technologische Wandel erfordert von Unternehmen wie der RZL Software GmbH eine kontinuierliche Anpassung und stellt sie vor strategischen sowie technischen Herausforderungen. Die Umstellung älterer Softwaresysteme bedeutet in der Regel einen enormen Ressourcenaufwand. Hinsichtlich der Effizienz kann eine gezielte Auswahl von Funktionalitäten auf Basis von Nutzerdaten wesentliche Steigerungen bewirken. Des Weiteren können etablierte Prozesse bei Anwendern des bestehenden Softwareprodukts ermittelt, beibehalten und optimiert werden.

Während es im Web-Umfeld bereits bestehende Standardlösungen zum Ermitteln von Nutzerdaten gibt, ist dies für komplexe Windows-Forms- und WPF-Rich-Client-Applikationen nicht der Fall. Daher beschäftigt sich diese Bachelorarbeit mit der Entwicklung eines Aktivitäts-Tracking-Frameworks, das für WPF- und Windows-Forms-Anwendungen flexibel eingesetzt werden kann. Dieses Framework soll es ermöglichen, Verhaltensdaten an das Unternehmen zu senden, ohne gravierende Änderungen am bestehenden System vorzunehmen und dabei die Benutzung des Programms nicht zu beeinträchtigen.

Die Arbeit beginnt mit dem Vergleich verschiedener Technologien und dem Erarbeiten der Grundlagen. Darauf aufbauend wird das Konzept für das Framework erstellt. Anschließend wird das System mittels C\# umgesetzt und zu Testzwecken jeweils in eine WPF- und eine Windows-Forms-Applikation integriert. Um die Funktionsfähigkeit zu demonstrieren, wird eine Konfiguration erstellt und aus dem bestehenden System entsprechende Daten ermittelt. Diese Daten bilden zugleich die Basis für die exemplarische Beantwortung von Fragen, die sich aufgrund des Optimierungsproblems ergeben haben.

\end{german}		
\chapter{Abstract}


\begin{english} %switch to English language rules
Rapid technological change requires companies such as RZL Software GmbH to continuously adapt, presenting them with strategic and technical challenges. Migrating older software systems usually requires enormous resources. In terms of efficiency, a selective choice of functionalities based on pseudonymised user data can lead to significant improvements. Furthermore, established processes can be identified, maintained and optimised by users.\\
\\
While standard solutions for this problem already exist on the web, this is not the case for complex Windows Forms and WPF rich client applications. This thesis therefore focuses on the development of an activity tracking framework that can be used as flexibly as possible for WPF and Windows Forms applications. This framework should make it possible to send behavioural data to the company without making any major changes to the existing system and without impairing the use of the programme.\\
\\
The thesis begins with a comparison of different technologies. The system design is then created based on these fundamentals. The system is then implemented using C\# and integrated into a WPF and a Windows Forms application for testing purposes. To demonstrate its functionality, a configuration is created and the relevant data is determined from the existing system. This data also forms the basis for answering the questions that have arisen as a result of the optimisation problem.
\end{english}



%%%-----------------------------------------------------------------------------
\mainmatter                             % Hauptteil (ab hier arab. Seitenzahlen)
%%%-----------------------------------------------------------------------------

\chapter{Einleitung}
\label{cha:einleitung}

\section{Hintergrund und Motivation}
\label{sec:motivation}

\section{Ziele der Arbeit}
\label{sec:goal}

\section{Vorgehensweise}
\label{sec:doing}


\chapter{Grundlagen}
\label{cha:grundlagen}
Um die Nachvollziehbarkeit der folgenden Inhalte zu gewährleisten, bereitet dieses Kapitel zunächst die Grundlagen auf. Es stellt vergleichbare Technologien vor und diskutiert verschiedene Ansätze zum Tracking von Nutzerverhalten. Darüber hinaus erläutert es zwei weitere Patterns, die speziell in der UI-Entwicklung zum Einsatz kommen und für das betrachtete System von Bedeutung sind.

\section{Vergleichbare Anwendungen}
\label{sec:similar_applications}
Speziell im Webbereich stehen bereits einige Frameworks und Systeme zur Verfügung, die es einfach und effizient machen, Nutzerverhalten aufzuzeichnen. Web und native Technologien unterscheiden sich zwar in vielen Aspekten, dennoch lassen sich einige Konzepte übertragen. Zwei der größten Vertreter von OpenTelemetry und Google sind daher in diesem Abschnitt beschrieben. Zu erwähnen ist, dass der Fokus auf der Datenermittlung der Systeme liegt, da die Analyse unabhängig von der Technologie anwendbar ist und die Problemstellung auf der Datenerfassung liegt.

\subsection{Google Analytics}
\label{subsec:google_analytics}

Google Analytics ist ein Tool, das auf der Technologie der Urchin Software Corporation basiert, die 2005 von Google übernommen wurde \cite{google2005urchin}. Ziel von Google Analytics ist es, Website-Betreibern mithilfe gesammelter Daten einen Überblick über die Nutzung ihrer Seiten zu verschaffen. Die gewonnenen Informationen sollen Unternehmen dabei unterstützen, ihre Marketingmaßnahmen gezielt zu steuern und deren Wirkung zu steigern.

\subsubsection{Technologie Überblick}
Die von Google bereitgestellte Abbildung \ref{fig:googel_analytics_architecture} zeigt die Architektur von Google Analytics und gibt einen Überblick über die verschiedenen Nutzungsmöglichkeiten. Wie zu Beginn des Abschnitts \ref{sec:similar_applications} beschrieben, ist Google einer der Vertreter, die eine flexible Schnittstelle zur Bereitstellung der Daten anbieten. In der Abbildung \ref{fig:googel_analytics_architecture} werden daher beispielhaft die Nutzungsmöglichkeiten über Web-Client, Server und Mobile aufgezeigt.
Wesentlich ist jedoch, dass die Daten in der im "Measurement Protocol" \cite{google_developers_sending_events} beschriebenen Form übermittelt werden. Die Auswertung der gesammelten Daten über die dargestellte Google Analytics-Benutzeroberfläche wird in dieser Arbeit nicht behandelt. Dennoch stellt Google für die drei genannten Plattformen Frameworks bereit, die eine automatische und vereinfachte Datensammlung ermöglichen und für diese Arbeit von Relevanz sind.

\begin{figure}[h]
\centering
\includegraphics[width=0.95\textwidth]{2_Architektur_Google_Analytics}
\caption{Architektur Google Analytics. Quelle \cite{google_developers_sending_events}}
\label{fig:googel_analytics_architecture}
\end{figure}

\subsubsection{Technologie für Webseiten}
Einer der in Abbildung \ref{fig:googel_analytics_architecture} dargestellten Datenquellen ist eine Webseite. 
Für deren Integration stellt Google JavaScript-Code bereit, der lediglich in die Webseite eingefügt werden muss. 
Sobald der Client dieses Skript ausführt, lädt es automatisch weitere Skripte von Google nach, welche die Ausführung vordefinierter und benutzerdefinierter Kommandos ermöglichen. 
Über diese Kommandos werden die Daten gemäß dem \textit{Measurement Protocol} \cite{google_developers_sending_events} an Google übermittelt. 
Dadurch ergibt sich folgende Datenhierarchie:

\begin{enumerate}
    \item \textbf{Hit}: Ein einzelner Datensatz, beispielsweise ein Ereignis wie ein Button-Klick.
    \item \textbf{Session}: Eine Sammlung mehrerer Ereignisse, die während einer Sitzung auftreten.
    \item \textbf{Property}: Fasst alle Hits zusammen, die mit derselben Property-ID gekennzeichnet sind (z.\,B. alle Daten einer Webseite).
    \item \textbf{View}: Repräsentiert eine bestimmte Ansicht oder einen gefilterten Ausschnitt der Daten einer Property.
\end{enumerate}

Die Daten können auf verschiedene Weise übermittelt werden. Eine dieser Varianten ist die Ausführung des in Programm \ref{prog:send_command} dargestellten JavaScript-Codes. Wie zu erkennen ist, werden die Daten in unterschiedliche Kategorien unterteilt, um spätere Filterungen und Abfragen zu erleichtern.

\begin{program}[H]
\begin{JsCode}
ga('send', 'social', 'network', 'action', 'target');
\end{JsCode}
\caption{Sende-Kommando in Google Analytics}
\label{prog:send_command}
\end{program}

Um den Quellcode übersichtlich zu halten und das Tracking flexibel zu gestalten, bietet Google zusätzlich den sogenannten \textit{Tag Manager} an. 
Tags stellen hierbei Code- oder HTML-Elemente dar, die je nach definierten Bedingungen (Triggern) automatisch in die Webseite eingefügt werden. 
Diese Tags lesen vordefinierte Objekte (Variablen) aus und übermitteln die Daten über die von Google Analytics bereitgestellte Schnittstelle. 
Trigger können beispielsweise benutzerdefinierte oder automatisch registrierte Ereignisse sein. 
Die Interaktion der Komponenten erfolgt über ein \texttt{dataLayer}-Objekt, in das Ereignisse, Metadaten und weitere Informationen eingetragen werden. 
Trigger, Variablen und Tags können über eine entsprechende Benutzeroberfläche konfiguriert werden. 
Für die beschriebenen Komponenten ergeben sich somit folgende Rollen:

\begin{itemize}
    \item \textbf{Trigger:} Lösen Tags aus, sobald bestimmte Ereignisse eintreten.
    \item \textbf{Tags:} Enthalten den Code zur Datenerfassung aus Variablen.
    \item \textbf{Variablen:} Dienen als Datenquellen und beziehen Informationen aus dem \texttt{dataLayer} oder dem DOM.
    \item \textbf{Events:} Werden automatisch aus dem DOM generiert oder manuell durch JavaScript-Code ausgelöst.
\end{itemize}

Der dargestellte Ablauf bezieht sich auf Universal Analytics. Sowohl die Funktionsweise als auch die Konfiguration werden im Buch von Weber \cite{weber2015practical} ausführlich beschrieben. Die Weiterentwicklung von Universal Analytics ist Google Analytics 4 (GA4). Für diese Arbeit ist jedoch die Betrachtung von Universal Analytics ausreichend, da es sich um vergleichbare Konzepte handelt.

\subsection{OpenTelemetry}
\label{subsec:open_telemetry}
Zur Erfassung von Metriken, Logs und Traces in Anwendungen und somit zur Erreichung einer besseren Observability bietet sich \cite[OpenTelemetry]{opentelemetry_what_is} an. OpenTelemetry ist ein Open-Source-Projekt, das eine einheitliche Erfassung, Verarbeitung und Weiterleitung von Telemetriedaten unabhängig vom jeweiligen Anbieter ermöglicht. Dafür stellt OpenTelemetry eine umfangreiche Sammlung von SDKs für verschiedene Programmiersprachen bereit. Auf diese Weise können selbst Anwendungen, die aus unterschiedlichen Technologien bestehen, ganzheitlich instrumentiert und überwacht werden.

\subsubsection{Aufbau von OpenTelemetry}
OpenTelemetry besteht im Wesentlichen aus drei Komponenten: der Auto-Instrumentierung, der SDK-basierten manuellen Instrumentierung sowie dem sogenannten Collector.  
Die Auto-Instrumentierung erfolgt über einen Agenten, der mit der jeweiligen Laufzeitumgebung verbunden ist und ohne Codeänderungen Metriken, Logs oder Traces erfassen kann.  
Der Collector sammelt alle von den verschiedenen Quellen erzeugten Telemetriedaten, verarbeitet sie (z. B. durch Aggregation oder Filterung) und exportiert sie anschließend an externe Systeme. Über sogenannte Exporter können die Daten an verschiedene Analyse- und Visualisierungstools wie Azure Monitor, Prometheus, OpenSearch oder Grafana weitergeleitet werden.

\subsubsection{Zero-code basierte Instrumentierung}
OpenTelemetry unterstützt – ähnlich wie \ref{subsec:google_analytics} Google Analytics – auch eine automatische Erkennung und Erfassung von Telemetriedaten. Für diese sogenannte {Auto-Instrumentierung} sind keine Änderungen am Anwendungscode notwendig. Stattdessen wird ein Agent genutzt, der zur Laufzeit mit der Zielanwendung interagiert und den Code automatisch instrumentiert. Diese Vorgehensweise kann, wie in Abschnitt \ref{subsec:autogenerated_code} erläutert, beispielsweise durch Bytecode- oder IL-Manipulation erfolgen. Der genaue Mechanismus hängt dabei von der verwendeten Technologie und Laufzeitumgebung (z. B. Java VM, .NET CLR) ab.

\subsubsection{Datenermittlung mit OpenTelemetry}
Im Gegensatz zu Google Analytics \ref{subsec:google_analytics} unterscheidet OpenTelemetry zwischen drei Arten von Telemetriedaten: Metriken, Traces und Logs.

\paragraph{Metriken} sind numerische Messwerte, die mit vordefinierten Messinstrumenten wie Counter, Histogramm oder Gauge erfasst werden. Diese Messinstrumente können über das OpenTelemetry-SDK im Anwendungscode instanziiert und für benutzerdefinierte Messpunkte genutzt werden.

\paragraph{Traces} beschreiben den Ausführungspfad einer Operation und bestehen aus mehreren Teilschritten, den sogenannten \emph{Spans}. Ein Span stellt dabei eine logisch abgegrenzte Aktion dar, etwa einen einzelnen Datenbankzugriff oder eine HTTP-Anfrage. Durch die Verkettung mehrerer Spans entsteht ein vollständiger Trace, der einen gesamten Ablauf abbilden kann, z.\,B. das Erstellen eines Produkts bis zum Hinzufügen in den Warenkorb.  
Für Web-Technologien, Datenbankzugriffe oder HTTP-Kommunikation existiert bereits eine weitgehend automatische Erkennung solcher Traces. Um Traces manuell zu integrieren, gibt es je nach verwendetem SDK die Möglichkeit, einen Span am Beginn einer Aktion zu starten und am Ende wieder zu schließen. Dazwischen können beliebige Mess- oder Beobachtungspunkte mit einer Beschreibung innerhalb des Spans aufgezeichnet werden. Die Verknüpfung der einzelnen Spans zu einem Trace erfolgt anschließend im Collector über eine eindeutige Trace-ID.

\paragraph{Logs} sind ein zentraler Bestandteil vieler Systeme. OpenTelemetry bietet die Möglichkeit, strukturierte und unstrukturierte Logdaten zentral zu erfassen und für eine nachgelagerte Verarbeitung bereitzustellen. Dabei können sowohl bestehende Logs integriert als auch neue, speziell für Observability-Zwecke erzeugte Logs verwendet werden.

Über diese Varianten von Telemetriedaten ermöglicht OpenTelemetry, ein breites Spektrum an Informationen zu erfassen und flexibel zu verarbeiten, wie in der Online-Dokumentation \cite{opentelemetry_what_is} ausführlich beschrieben ist.

\section{Mögliche Ansätze für Aktivitäts-Tracking}
\label{sec:solutions_tracking}

\subsection{Proxy Server}
\label{subsec:proxy_server}

\subsection{Externes JavaScript}
\label{subsec:external_js}

\subsection{Automatisch generierter Code}
\label{subsec:autogenerated_code}

\subsection{Aufgabendelegation}
\label{subsec:task_delegation}

\section{Softwaremuster für UI Frameworks}
\label{subsec:patterns}

\subsection{MVVM}
\label{subsec:mvvm}

\subsection{MVP}
\label{subsec:mvp}







\chapter{Technologie und Werkzeuge}
\label{cha:technologie_werkzeuge}

\section{Visual Studio}
\label{sec:visual_studio}

\section{.NET}
\label{sec:dotnet}

\subsection{WPF}
\label{subsec:WPF}

\subsection{Windows Forms}
\label{subsec:Winforms}

\subsection{C\#}
\label{subsec:csharp}

\subsection{Assemblies}
\label{subsec:assemblies}

\subsection{Asynchrone Programmierung}
\label{subsec:async}

\subsection{Reflection}
\label{subsec:reflection}


\chapter{Bestehendes System}
\label{cha:basis}
Da die Bachelorarbeit bereits in ein bestehendes System integriert wird, gibt es Möglichkeiten, bereits vorhandenen Code wiederzuverwenden. Um die Implementierung (Kapitel \ref{cha:implementierung}) nachvollziehen zu können, werden in diesem Kapitel einzelne bestehende Konzepte vorgestellt, die dabei helfen, das Framework später zu integrieren.

\section{MVVM spezifische Erweiterungen}
\label{sec:mvvm_extensions}
Grundsätzlich basiert das WPF-System (siehe Unterabschnitt \ref{subsec:WPF}) auf dem in Unterabschnitt \ref{subsec:mvvm} beschriebenen MVVM-Muster. Würde die gesamte Logik jedoch direkt in die Vererbungshierarchie der ViewModels integriert, würde dies die Flexibilität erheblich einschränken und die Komplexität deutlich erhöhen. Darüber hinaus können Datenbindungen komplexere Szenarien nicht immer korrekt abbilden. Daher wurde das MVVM-Muster an die spezifischen Anforderungen der bestehenden Anwendung angepasst.

\subsection{Presenter}
Datenbindungen ermöglichen zwar einen relativ einfachen Datenaustausch zwischen View und ViewModel, doch treten häufig Situationen auf, in denen Daten das Verhalten der Ansicht beeinflussen oder Aktionen im ViewModel ausgelöst werden müssen, die sich über einfache Kommandos\footnote{Kommandos \cite{wpf_commanding_overview} sind ein Konzept, bei dem ein Objekt erstellt wird, das eine Aufgabe ausführt, sobald es getriggert wird. Dieses Objekt kann über Datenbindung ausgetauscht werden.} nicht mehr abbilden lassen.

In der betreffenden WPF-Anwendung kommen daher Presenter zum Einsatz, die die Kommunikation zwischen ViewModel und View sowohl auf Basis von Daten als auch über Events ermöglichen. Dem Presenter sind sowohl die View als auch das ViewModel bekannt, jedoch können weder View noch ViewModel auf den Presenter zugreifen.

Der Erzeugungsprozess des MVVM-Konstrukts mit Presenter sowie die beschriebenen Zusammenhänge sind in Abbildung \ref{fig:mvvm_with_presenter} dargestellt. Die dort abgebildete ViewModel Region ist ein Control, das die Darstellung eines ViewModels mithilfe der entsprechenden View vornimmt. Die ViewModel Region verwendet eine Fabrik (siehe Factory-Muster \cite{gamma1995design}), um die entsprechenden Presenter zu erzeugen. Die dazugehörigen Informationen, welche View und welcher Presenter für ein bestimmtes ViewModel erzeugt werden sollen, entnimmt die Fabrik der MVVM-Config\footnote{Die MVVM-Config ist eine Konfiguration, die die Zuordnung von ViewModel, View, Presenter und deren Erweiterungen festlegt.}. 

Die grünen Pfeile in der Abbildung \ref{fig:mvvm_with_presenter} veranschaulichen den Datenaustausch über Datenbindungen sowie die Events und Daten, die über den Presenter weitergeleitet werden können, wenn beispielsweise kein Kommando oder keine verwendbare Datenbindung zur Verfügung steht.

\begin{figure}[H]
    \centering
    \includegraphics[width=0.8\textwidth]{4_Presenter_MVVM_Extension}
    \caption{Erstellungsprozess und Zusammenhänge des MVVM-Musters mit Presenter.}
    \label{fig:mvvm_with_presenter}
\end{figure}

\subsection{ViewModel Erweiterungen}
\label{subsec:viewmodel_extensions}
ViewModel-Erweiterungen erweitern ein ViewModel um zusätzliche Funktionalitäten, wie z.B. automatisches Speichern oder Datenvalidierung. Diese Erweiterungen werden für jedes ViewModel in einer Liste gespeichert und gemeinsam mit dem ViewModel initialisiert. Eine ViewModel-Erweiterung erhält Zugriff auf das ViewModel, das sie erweitert, und kann über dessen Eigenschaften auch für die Datenbindung an die Ansicht verwendet werden. ViewModel-Erweiterungen müssen dem ViewModel bereits während des Konstruktionsprozesses hinzugefügt werden.

\subsection{Presenter Erweiterungen}
\label{subsec:presenter_extensions}
Presenter-Erweiterungen sind analog zu den zuvor beschriebenen ViewModel-Erweiterungen, erweitern jedoch den Presenter. Sie werden über eine Fabrik in der MVVM-Config definiert. Dort können Presenter-Erweiterungen gemeinsam mit dem jeweiligen Presenter, der View, dem ViewModel und der zugehörigen ViewModel-Erweiterung erzeugt werden. Presenter-Erweiterungen besitzen keine festgelegte Struktur und müssen bei ihrer Erstellung initialisiert werden. Wie die ViewModels selbst werden sie im entsprechenden Presenter gehalten.

\section{Windows Forms spezifische Erweiterungen}
\label{sec:mvp_extensions}
Wie auch in WPF (siehe Unterabschnitt \ref{subsec:WPF}) gibt es für Windows Forms (siehe Unterabschnitt \ref{subsec:Winforms}) einige Erweiterungen, welche die Entwicklung erleichtern und für die Integration des Frameworks in die Anwendung notwendig sind.

\subsection{Eingebettete Presenter}
\label{subsec:embedded_presenter}
In der mit Windows Forms umgesetzten Anwendung wird das MVP-Muster (siehe Unterabschnitt \ref{subsec:mvp}) angewendet. Dennoch gibt es Aufgaben, die sich wiederholen und sich gut aus dem Hauptpresenter herauslösen lassen, um eine höhere Kohäsion zu erreichen. Für solche Aufgaben werden eingebettete Presenter eingesetzt. Diese werden in einem Initialisierungsschritt im Hauptpresenter erzeugt und erhalten Zugriff auf diesen. Dadurch können typische Aufgaben wie Validierung, Speichern oder Ähnliches ausgelagert sowie flexibel ausgetauscht oder ergänzt werden. Nach ihrer Erzeugung werden eingebettete Presenter unmittelbar initialisiert. Wie auch bei ViewModel- und Presenter-Erweiterungen wird nach dem Prinzip Delegation vor Vererbung gehandelt \cite{eilebrecht2010patterns}.

\subsection{UI Accessor}
\label{subsec:ui_accessor_component}
Um auf Events von Controls zugreifen zu können, wird der UI Accessor verwendet. Dieser abstrahiert verschiedene Events mit gleicher Bedeutung zu einem gemeinsamen, standardisierten Event und ermöglicht dadurch die Registrierung von Handlern auf diese vereinheitlichten Ereignisse. Dies vereinfacht die Handhabung von Benutzerinteraktionen, da bei der Registrierung kein Wissen über das konkrete Event oder den zugehörigen Event-Handler erforderlich ist. Die abstrahierten Events werden über den Kommando-Mapper als sogenannte Kommandos bereitgestellt und können über den Presenter registriert oder deregistriert werden.

Da es in Windows Forms viele unterschiedliche Events mit gleicher Bedeutung gibt, spielt diese Abstraktion eine wichtige Rolle. Darüber hinaus können auch benutzerdefinierte UserControls mit eigenen Events zentral berücksichtigt werden.

\section{Visual Tree Helper}
\label{sec:visual_tree_helper}
Sowohl in Windows Forms (siehe Unterabschnitt \ref{subsec:Winforms}) als auch in WPF (siehe Unterabschnitt \ref{subsec:WPF}) wird ein Baum aus visuellen Elementen nach dem Prinzip des Composite-Muster aufgebaut (siehe Muster GoF \cite{gamma1995design}).

Gerade für die Registrierung von Events ist das Auffinden bestimmter Controls in diesem Baum entscheidend. Daher gibt es sowohl für WPF als auch für Windows Forms einen Visual Tree Helper, der Elemente rekursiv im Baum findet.

Eine klare Unterscheidung zwischen dem logischen und dem visuellen Baum ist in WPF \cite{microsoft_trees_in_wpf} wesentlich und wird in Abbildung \ref{fig:logical_visual_tree} veranschaulicht. Während der visuelle Baum ausschließlich die tatsächlich sichtbaren Elemente einer Benutzeroberfläche umfasst, beschreibt der logische Baum die hierarchische Struktur der Ansicht, einschließlich jener Objekte, die beispielsweise als Container, Vorlage oder Datenprovider für sichtbare Elemente dienen.

\begin{figure}[H]
    \centering
    \includegraphics[width=0.5\textwidth]{4_Logical_Visual_Tree}
    \caption{Visueller und logischer Elementbaum.}
    \label{fig:logical_visual_tree}
\end{figure}

\section{AssemblyRegisterResolver}
\label{sec:assembly_resolver}
Für die Konfiguration in Abschnitt \ref{sec:configuration_concept} müssen bestimmte Assemblies (siehe Unterabschnitt \ref{subsec:assemblies}) gefunden werden, die Klassen enthalten, welche ein bestimmtes Interface implementieren.

Da dies insbesondere bei größeren Anwendungen zu Performanceproblemen führen kann, werden alle relevanten Assemblies bereits während des Build-Prozesses auf die Implementierung entsprechender Interfaces überprüft. Die dabei gefundenen Assemblies werden anschließend registriert und in einem Zwischenspeicher abgelegt.

Bei späteren Suchvorgängen nach Klassen, die ein bestimmtes Interface implementieren, müssen dadurch nicht mehr alle Assemblies durchsucht werden, sondern nur diejenigen, die tatsächlich relevant sind. Dies reduziert den Suchaufwand und verbessert die Performance der Anwendung.
\chapter{Konzept}
\label{cha:konzept}
Der Aufbau und das Systemdesign des Aktivitäts-Tracking-Frameworks bilden einen zentralen Bestandteil dieser Arbeit. In diesem Kapitel werden die einzelnen Komponenten beschrieben, die für das Tracking von Aktivitätsdaten erforderlich sind. Die praktische Umsetzung des Konzepts erfolgt anschließend in Kapitel \ref{cha:implementierung}.

\section{Systemarchitektur}
Bevor die einzelnen Komponenten im Detail erläutert werden, beschreibt dieser Abschnitt die Zusammenarbeit und die übergeordnete Struktur der Teilbereiche und Systemkomponenten.

\subsection{Komponenten und Aufbau}

\subsubsection{Teilbereiche und Komponenten}
\label{sec:system_design}
Das Tracking-Framework besteht aus sechs Teilbereichen, die gemeinsam den gesamten Funktionsumfang des Systems abdecken.

\begin{itemize}
    \item Systemkonfiguration
    \item Trackingkonfiguration
    \item Daten- und Aktionsermittlung
    \item Filterung und Extraktion
    \item Vermittlung und Ablaufsteuerung
    \item Datenaustausch und Zwischenspeicherung
\end{itemize}

Im Rahmen der Implementierung (siehe Kapitel \ref{cha:implementierung}) wurden diese Teilbereiche in eigenständige Komponenten unterteilt, sodass alle Aufgaben aus diesen Bereichen berücksichtigt werden. Bestimmte Teilbereiche, wie beispielsweise der Datenaustausch und die Zwischenspeicherung, wurden dabei auf mehrere Komponenten verteilt.

\subsubsection{Struktur der Komponenten}
Abbildung \ref{fig:system_design_components} zeigt, wie die einzelnen Komponenten im System zusammenarbeiten. Das zentrale Element bildet der {Tracking-Manager}, über den alle Komponenten miteinander verbunden sind. Jede Komponente verfügt über eine klar definierte Schnittstelle, die den Zugriff ermöglicht. Nur der Tracking-Manager kennt die einzelnen Komponenten, während diese untereinander vollständig entkoppelt und unabhängig voneinander agieren. Die einzige Komponente, die von mehreren Komponenten genutzt wird, ist die Systemkonfiguration. Sie wird vom Tracking-Manager verwaltet und an die jeweiligen Komponenten weitergegeben.

Diese Struktur sorgt für eine hohe Flexibilität. Änderungen, die von der Systemumgebung abhängen, können vorgenommen werden, ohne andere Teile des Frameworks zu beeinflussen. So lässt sich beispielsweise die Art der Datenverarbeitung anpassen, ohne dass der übrige Aufbau geändert werden muss.

\begin{figure}[H]
    \centering
    \includegraphics[width=0.8\textwidth]{5_Systemdesign_Components_Tracking}
    \caption{Zusammenhänge der Komponenten im Tracking-Framework}
    \label{fig:system_design_components}
\end{figure}

Die lose Kopplung besteht nicht nur zwischen den internen Komponenten des Frameworks, sondern auch zwischen der Anwendung und der Datensammlung (siehe Abbildung \ref{fig:system_design_components}). Welche Daten erfasst werden und an welcher Stelle die Erhebung erfolgt, wird vollständig durch das Framework gesteuert.

Die Anwendung selbst muss im Wesentlichen lediglich die erforderlichen Daten über eine definierte Schnittstelle bereitstellen. Dadurch lässt sich das System flexibel in unterschiedlichen Technologien einsetzen, beispielsweise in WPF (siehe Unterabschnitt \ref{subsec:WPF}) oder Windows Forms (siehe Unterabschnitt \ref{subsec:Winforms}).

\subsection{Kommunikation zwischen Komponenten}

\subsubsection{Kommunikationsgrundlage}
Damit ein Informationsaustausch zwischen den Komponenten möglich ist, wird eine gemeinsame Kommunikationsbasis benötigt. Diese Grundlage besteht aus Objekten, die die gemeinsame Sprache für den Datenaustausch definieren.  
Konzeptionell wird zwischen fünf Informationskategorien unterschieden:

\begin{itemize}
    \item \textbf{Trackingaktionen}: Repräsentieren auftretende Ereignisse innerhalb der Anwendung.
    \item \textbf{Trackingdaten}: Umfassen Daten, die auf Basis einer aufgetretenen Aktion erfasst werden.
    \item \textbf{Extraktionsaufgaben}: Beschreiben, welche Informationen aus Trackingaktionen und Trackingdaten ermittelt werden sollen, und stellen Metadaten für die resultierenden Extraktionsdaten bereit.
    \item \textbf{Trackingaufgaben}: Legen fest, welche Daten für die Weiterverarbeitung ermittelt und verarbeitet werden dürfen.
    \item \textbf{Extraktionsdaten}: Enthalten die aus Trackingaktionen und Trackingdaten extrahierten Informationen, die anschließend weiterverarbeitet oder übermittelt werden können.
\end{itemize}

Die einzelnen Kategorien können verschiedene Objekttypen enthalten, die je nach Anwendungsfall unterschiedlich ausgestaltet sind. Diese Objekte bilden einen grundsätzlich unveränderlichen Standard, wobei Extraktionsdaten, Trackingaktionen und Trackingdaten durch benutzerdefinierte Objekte erweitert werden können. Damit die Daten vom Framework verarbeitet werden können, müssen sie, ähnlich wie bei Google Analytics (siehe Unterabschnitt \ref{subsec:google_analytics}), einer festgelegten Struktur entsprechen.

\subsubsection{Datenübertragungswege}
Der Datenaustausch zwischen den Komponenten kann entweder direkt über Rückgabewerte von Funktionen oder über sogenannte Datenkanäle erfolgen.  
Ein Datenkanal ist im Rahmen dieser Arbeit als ein Konstrukt zu verstehen, über das Daten an eine unbekannte oder externe Komponente weitergegeben werden können.  
Durch die Systemkonfiguration (siehe Abschnitt \ref{sec:integration_concept}) kann die Anwendung verschiedene Wege der Veröffentlichung oder des Bezugs von Daten anbieten. Auf diese Weise bleibt das System flexibel gegenüber unterschiedlichen Datenquellen und Zielen.

\subsubsection{Ablauf der Kommunikation}
Der Ablauf der Kommunikation erfolgt über die zuvor beschriebenen Datenobjekte und wird im in Abbildung~\ref{fig:sequence_diagram_communication_components} dargestellten Sequenzdiagramm veranschaulicht.  
Wie dort gezeigt, erzeugt die Anwendung zunächst alle benötigten Komponenten. Die Reihenfolge ergibt sich aus den jeweiligen Abhängigkeiten, wobei die genaue Erstellungsreihenfolge zwischen Datenermittlung (Abschnitt \ref{sec:data_collection_concept}), Verarbeitung (Abschnitt \ref{sec:data_extraction_concept}) und Trackingkonfiguration (Abschnitt \ref{sec:configuration_concept}) nicht entscheidend ist.  

Nach der Erzeugung wird der Tracking-Manager initialisiert. Während dieser Initialisierung wird auch die Systemkonfiguration (siehe Abschnitt \ref{sec:integration_concept}) übergeben. Anschließend initialisiert der Tracking-Manager die Trackingkonfiguration, die wiederum die benötigten Assemblies für die Konfiguration ermittelt. Danach kann der Tracking-Manager die Aufgaben für das Tracking anfordern. Diese Aufgaben werden bei der Initialisierung der Komponenten für Datenermittlung sowie für Verarbeitung und Extraktion berücksichtigt.

Nach Abschluss der Initialisierung kann die Anwendung sogenannte Data Provider registrieren. Dabei entscheidet die Komponente zur Datenermittlung, ob ein Provider einen entsprechenden Datenkanal zur Übertragung von Daten erhält. Über diesen Kanal können sowohl Aktivitätsdaten als auch nachgelieferte Daten gesendet werden.

Die übergebenen Daten werden anschließend in der Extraktionskomponente asynchron zur Anwendung empfangen und weiterverarbeitet.  

Wenn die Anwendung das Ausliefern der Daten anstößt, fordert der Tracking-Manager die Extraktionsdaten an und übergibt sie über den in der Systemkonfiguration bereitgestellten Datenkanal an die entsprechende Zielkomponente.

\begin{figure}[H]
    \centering
    \includegraphics[width=1.07\textwidth]{5_Sequence_Diagram_Components_Communication}
    \caption{Ablauf der Kommunikation zwischen den Komponenten}
    \label{fig:sequence_diagram_communication_components}
\end{figure}

\section{Tracking-Konfiguration}
\label{sec:configuration_concept}
Wie bei Google Analytics (siehe Unterabschnitt \ref{subsec:google_analytics}) und OpenTelemetry (siehe Unterabschnitt \ref{subsec:open_telemetry}) verfügt auch dieses Framework über eine Konfiguration, in der festgelegt wird, welche Daten getrackt werden sollen. Für das hier vorgestellte Framework ist eine hybride Lösung vorgesehen, die sowohl Online-Konfigurationen, wie bei Google Analytics, als auch Code-basierte Konfigurationen, wie bei OpenTelemetry, ermöglicht. In dieser Arbeit wird jedoch ausschließlich die zweite Variante konzipiert und umgesetzt.

\subsection{Aufbau der Konfigurationskomponente}
Der Aufbau der Konfigurationskomponente ist in Abbildung \ref{fig:configuration_component} dargestellt. Diese Komponente besteht aus einem Manager, der sowohl für die Initialisierung als auch für die Bereitstellung der Aufgaben verantwortlich ist. Der Manager agiert dabei als Slave des Hauptmanagers.

Das Erstellen der Konfiguration erfolgt durch sogenannte Configuration Builder, die zuvor über eine Fluent API \footnote{Unter einer Fluent API oder auch einem Fluent Interface \cite{Fowler2005FluentInterface} versteht man eine Methodik des Method-Chaining, bei der Konfigurationen in einer flüssigen, satzähnlichen Syntax formuliert werden können.} definiert werden. Optional soll die Komponente so erweitert werden können, dass künftig auch ein Konfigurationsserver angebunden werden kann. Im Rahmen dieser Arbeit bleibt dies jedoch unberücksichtigt.

\begin{figure}[H]
    \centering
    \includegraphics[width=0.8\textwidth]{5_Configuration_Component}
    \caption{Aufbau der Konfigurationskomponente}
    \label{fig:configuration_component}
\end{figure}

\subsection{Konfigurationsmöglichkeiten und Fluent API}

\subsubsection{Kategorien von Informationen}
Um die ursprünglich definierten Fragen (siehe Unterabschnitt \ref{subsec:initial_questions}) beantworten zu können, müssen bestimmte Informationen gesammelt werden. Diese lassen sich in folgende Kategorien einteilen:

\begin{itemize}
    \item \textbf{Metriken:} Zahlenbasierte Daten, wie sie bereits im Zusammenhang mit OpenTelemetry (siehe Unterabschnitt \ref{subsec:open_telemetry}) beschrieben wurden. Beispiele hierfür sind die Ladezeit einer Ansicht oder die Anzahl der Öffnungen einer bestimmten Ansicht.
    \item \textbf{Daten:} Informationen, die beispielsweise aus einer Ansicht extrahiert werden können, etwa die Anzahl der Einträge in einer Liste.
    \item \textbf{Nutzung:} Informationen, die Aufschluss über das Nutzungsverhalten der Anwendung geben, beispielsweise wie häufig ein Shortcut in einer bestimmten Ansicht verwendet wird.
    \item \textbf{Workflows:} Informationen, die mit Traces in OpenTelemetry vergleichbar sind und einen Ablauf von aufeinanderfolgenden Aktionen darstellen.
\end{itemize}

\subsubsection{Aufbau der Fluent API}
Die zuvor kategorisierten Daten können weiter beschrieben werden, woraus sich ein Schema ableiten lässt, das für den Aufbau der Fluent API von zentraler Bedeutung ist.\\
\\
$Kontext \rightarrow Kategorie \rightarrow Kategorieoptionen$\\
\\
Daten stammen stets aus einem bestimmten Kontext und gehören zu einer der oben definierten Kategorien. Diese Daten können anschließend durch spezifische Optionen weiter unterteilt werden. Ein vereinfachter Ausschnitt der Fluent API nach diesem Schema wird in Abbildung~\ref{fig:configuration_component_fluent_api} als Syntaxdiagramm dargestellt. Mit der in der Abbildung gezeigten API ist es möglich, das Anfordern von Daten aus einer Ansicht zu konfigurieren.

\begin{figure}[H]
    \centering
    \includegraphics[width=\textwidth]{5_Configuration_Component_Fluent_API.pdf}
    \caption{Vereinfachter Ausschnitt der Fluent API Tracking-Konfiguration}
    \label{fig:configuration_component_fluent_api}
\end{figure}


\section{Daten- und Aktionsermittlung}
\label{sec:data_collection_concept}

\section{Filterung und Extraktion}
\label{sec:data_extraction_concept}

\section{Systemkonfiguration}
\label{sec:integration_concept}



\chapter{Implementierung}
\label{cha:implementierung}

\section{Konfiguration}
\label{sec:configuration_impl}

\section{Daten- und Aktionsermittlung}
\label{sec:data_collection_impl}

\section{Filterung und Extraktion}
\label{sec:data_extraction_impl}

\section{Integration WPF}
\label{sec:integration_wpf_impl}

\section{Integration Windows Forms}
\label{sec:integration_winforms_impl}
\chapter{Anwendung und Auswertung}
\label{cha:auswertung}

\section{Ermitteln von Verhaltensdaten}
\label{sec:testing}

\section{Auswertung und Schlussfolgerung}
\label{sec:results}

\chapter{Diskussion und Ausblick}
\label{cha:diskussion}
In diesem Kapitel wird die Bachelorarbeit noch einmal aufgearbeitet und sowohl die positiven als auch die negativen Merkmale werden hervorgehoben.
Anschließend wird dargestellt, wie diese Arbeit in Zukunft weiterverwendet werden kann und welche konkreten Weiterentwicklungsideen bestehen.

\section{Zielerreichung}
Das erarbeitete Konzept sowie die Implementierung können die zu Beginn der Arbeit formulierten Fragestellungen (siehe Unterabschnitt \ref{subsec:initial_questions}) beantworten und gehen in einigen Punkten sogar über den ursprünglichen Umfang hinaus.
Damit kann festgestellt werden, dass die gesetzten Ziele insgesamt erreicht wurden.

\section{Verbesserungspotential}
Die Arbeit ist im Laufe des Projekts mit zunehmender Erfahrung gewachsen. Dadurch gibt es einige Stellen, an denen im Nachhinein andere Lösungsansätze zu besseren Ergebnissen hätten führen können.

\subsection{Probleme mit Workflows}

\subsubsection{Zusätzlicher Aktionstyp}
Im Nachhinein wäre es sinnvoll gewesen, einen zusätzlichen Aktionstyp für das Wechseln zwischen Ansichten bereitzustellen. Aktuell besteht das Problem, dass es schwierig ist, den Beginn und das Ende eines Workflows eindeutig zu identifizieren.

\subsubsection{Nachgelagerte Aggregation von Workflows}
Workflows erzeugen eine enorme Menge an Daten, die insbesondere durch routete Events häufig nicht relevant sind. Ursprünglich war vorgesehen, diese Daten weiter zu filtern und bereits in aggregierter Form darzustellen. Es hat sich jedoch gezeigt, dass eine Aggregation schwierig ist, da sich die Workflows grundlegend unterscheiden. Daher wurde beschlossen, dass Darstellung und Verarbeitung nachgelagert erfolgen sollen und somit nur eine grundlegend Filterung statt findet.

\subsection{Konfigurationsduplikate}
Bei den konfigurierten Aufträgen wird versucht, Duplikate zu vermeiden.
Es gibt jedoch Fälle, insbesondere bei Daten, die aus einem Präsentationsmodell oder einer Ansicht stammen, in denen sich solche Duplikate nicht vollständig verhindern lassen. Dies liegt an der Designentscheidung, dass Ansicht und Präsentationsmodell für bestimmte Konfigurationen die gleiche Bedeutung haben, später jedoch nicht eindeutig zugeordnet werden kann, welche Ansicht zu welchem Präsentationsmodell gehört.Um dieses Problem zu lösen, wären Analyzer erforderlich, die diese Zusammenhänge zur Compile-Zeit automatisch erkennen könnten.

\subsection{Umfang des Projekts}
Das System wurde in dieser Arbeit so gut wie möglich dargestellt.
Allerdings ist der Umfang des tatsächlichen Programms deutlich größer, sodass im Rahmen dieser Arbeit nicht auf alle Punkte und Konzepte eingegangen werden konnte.
Daher wirkt die Arbeit an einigen Stellen etwas zu abstrakt.

\section{Erfahrungsgewinn}

\subsection{Programmierung mit C\#}
Da dieses Projekt sehr umfangreich war und dabei eine große Menge an C\#-Code (siehe Unterabschnitt \ref{subsec:csharp}) entwickelt wurde, konnten die Kenntnisse in dieser Programmiersprache deutlich vertieft werden.
Zudem wurde intensiv mit der Reflection-API (siehe Unterabschnitt \ref{subsec:reflection}) gearbeitet, die zwar bereits vor Beginn der Arbeit bekannt war, jedoch kein tieferes Verständnis dafür vorhanden war.

\subsubsection{Designmuster und Architektur}
Das Projekt hat gezeigt, wie wichtig der Einsatz von Designmustern ist und welchen Einfluss diese auf die Verständlichkeit und Wartbarkeit von Software haben.

Dabei konnte gelernt werden, wie entscheidend eine sorgfältige Planung und die Aufteilung des Softwaresystems in entkoppelte Einheiten ist.
Dies ermöglicht eine effizientere und fokussiertere Arbeit an den einzelnen Teilbereichen.

\section{Weitere Ausbaustufen des Frameworks}
Wie zuvor beschrieben, bestehen noch einige Probleme, die behoben werden können.
Darüber hinaus gibt es mehrere Ideen zur Weiterentwicklung des bestehenden Systems, die im Folgenden kurz beschrieben werden.

\subsubsection{Konfiguration über einen Online-Service}
Wie bereits in Unterabschnitt \ref{sec:configuration_concept} erwähnt, ist die Konfiguration derzeit ausschließlich über die Fluent API möglich.
In Systemen wie Google Analytics (siehe Unterabschnitt \ref{subsec:google_analytics}) erfolgt die Konfiguration hingegen online über den Tag Manager.

Da lange Releasezyklen die Flexibilität des Frameworks stark einschränken, wäre es eine sinnvolle Erweiterung, die Konfiguration über einen Online-Service bereitzustellen. Damit könnte die Konfiguration zusätzlich dynamisch angepasst und zentral verwaltet werden, ohne dass eine neue Programmversion ausgerollt werden muss.

\subsubsection{Visualisierung der Verhaltensdaten}
Die gesammelten Daten können zwar bereits übertragen und beispielsweise in einer Datenbank abgefragt werden. Zukünftig soll es jedoch möglich sein, Workflows als Graphen darzustellen oder die Daten mit Abbildungen von Ansichten zu verknüpfen, um eine bessere und schnellere, nicht-technische Übersicht über das Verhalten zu erhalten.


\section{Künftige Verwendung des Frameworks}
Für den praktischen Einsatz ist noch eine ausführliche Testphase erforderlich, in deren Rahmen Unit-Tests erstellt werden müssen. Zudem sollte das Tracking zunächst intern getestet werden, bevor es für alle Anwender*innen freigeschaltet wird.

Wenn das Framework korrekt funktioniert, wird es voraussichtlich weitere Anfragen zur Erweiterung der Ermittlungsmöglichkeiten geben, die anschließend umgesetzt werden können. Wobei das Framework bereits einen umfangreichen Funktionsbereich abdeckt.


%%%-----------------------------------------------------------------------------
\appendix                                                               % Anhang 
%%%-----------------------------------------------------------------------------

\include{anhang.tex}

%%%-----------------------------------------------------------------------------
\backmatter                          % Schlussteil (Quellenverzeichnis und dgl.)
%%%-----------------------------------------------------------------------------

\MakeBibliography % Quellenverzeichnis

\listoffigures
\addcontentsline{toc}{chapter}{Abbildungsverzeichnis}

%%%-----------------------------------------------------------------------------
% Messbox zur Druckkontrolle
%%%-----------------------------------------------------------------------------

\include{back/messbox}

%%%-----------------------------------------------------------------------------
\end{document}
%%%-----------------------------------------------------------------------------
