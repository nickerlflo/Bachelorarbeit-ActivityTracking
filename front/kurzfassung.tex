\chapter{Kurzfassung}

\begin{german}

Der schnelle technologische Wandel erfordert von Unternehmen wie der RZL Software GmbH eine kontinuierliche Anpassung und stellt sie vor strategische sowie technische Herausforderungen. Die Umstellung älterer Softwaresysteme bedeutet in der Regel einen enormen Ressourcenaufwand. Hinsichtlich der Effizienz kann eine selektive Auswahl von Funktionalitäten auf Basis pseudonymisierter Nutzerdaten wesentliche Steigerungen bewirken. Des Weiteren können etablierte Prozesse von Anwendern ermittelt, beibehalten und optimiert werden.\\
\\
Während es im Web bereits bestehende Standardlösungen für dieses Problem gibt, ist dies für komplexe Windows-Forms- und WPF-Rich-Client-Applikationen nicht der Fall. Daher beschäftigt sich diese Arbeit mit der Entwicklung eines Aktivitäts-Tracking-Frameworks, das für WPF- und Windows-Forms-Anwendungen möglichst flexibel eingesetzt werden kann. Dieses Framework soll es ermöglichen, Verhaltensdaten an das Unternehmen zu senden, ohne gravierende Änderungen am bestehenden System vorzunehmen und dabei die Benutzung des Programms nicht zu beeinträchtigen.\\
\\
Der Einstieg erfolgt mit dem Vergleich verschiedener Technologien. Daraufhin wird das Systemdesign auf Basis dieser Grundlagen erstellt. Anschließend wird das System mittels C\# umgesetzt und zu Testzwecken jeweils in eine WPF- und eine Windows-Forms-Applikation integriert. Um die Funktionsfähigkeit zu demonstrieren, wird eine Konfiguration erstellt und aus dem bestehenden System entsprechende Daten ermittelt. Diese Daten bilden zugleich die Basis für die Beantwortung der Fragen, die sich aufgrund des Optimierungsproblems ergeben haben.

\end{german}

