\chapter{Konzept}
\label{cha:konzept}
Der Aufbau und das Systemdesign des Aktivitäts-Tracking-Frameworks bilden einen zentralen Bestandteil dieser Arbeit. In diesem Kapitel werden die einzelnen Komponenten beschrieben, die für das Tracking von Aktivitätsdaten erforderlich sind. Die praktische Umsetzung des Konzepts erfolgt anschließend in Kapitel \ref{cha:implementierung}.

\section{Systemarchitektur}
Bevor die einzelnen Komponenten im Detail erläutert werden, beschreibt dieser Abschnitt die Zusammenarbeit und die übergeordnete Struktur der Teilbereiche und Systemkomponenten.

\subsection{Komponenten und Aufbau}

\subsubsection{Teilbereiche und Komponenten}
\label{sec:system_design}
Das Tracking-Framework besteht aus sechs Teilbereichen, die gemeinsam den gesamten Funktionsumfang des Systems abdecken.

\begin{itemize}
    \item Systemkonfiguration
    \item Trackingkonfiguration
    \item Daten- und Aktionsermittlung
    \item Filterung und Extraktion
    \item Vermittlung und Ablaufsteuerung
    \item Datenaustausch und Zwischenspeicherung
\end{itemize}

Im Rahmen der Implementierung (siehe Kapitel \ref{cha:implementierung}) wurden diese Teilbereiche in eigenständige Komponenten unterteilt, sodass alle Aufgaben aus diesen Bereichen berücksichtigt werden. Bestimmte Teilbereiche, wie beispielsweise der Datenaustausch und die Zwischenspeicherung, wurden dabei auf mehrere Komponenten verteilt.

\subsubsection{Struktur der Komponenten}
Abbildung \ref{fig:system_design_components} zeigt, wie die einzelnen Komponenten im System zusammenarbeiten. Das zentrale Element bildet der {Tracking-Manager}, über den alle Komponenten miteinander verbunden sind. Jede Komponente verfügt über eine klar definierte Schnittstelle, die den Zugriff ermöglicht. Nur der Tracking-Manager kennt die einzelnen Komponenten, während diese untereinander vollständig entkoppelt und unabhängig voneinander agieren. Die einzige Komponente, die von mehreren Komponenten genutzt wird, ist die Systemkonfiguration. Sie wird vom Tracking-Manager verwaltet und an die jeweiligen Komponenten weitergegeben.

Diese Struktur sorgt für eine hohe Flexibilität. Änderungen, die von der Systemumgebung abhängen, können vorgenommen werden, ohne andere Teile des Frameworks zu beeinflussen. So lässt sich beispielsweise die Art der Datenverarbeitung anpassen, ohne dass der übrige Aufbau geändert werden muss.

\begin{figure}[H]
    \centering
    \includegraphics[width=0.8\textwidth]{5_Systemdesign_Components_Tracking}
    \caption{Zusammenhänge der Komponenten im Tracking-Framework}
    \label{fig:system_design_components}
\end{figure}

Die lose Kopplung besteht nicht nur zwischen den internen Komponenten des Frameworks, sondern auch zwischen der Anwendung und der Datensammlung (siehe Abbildung \ref{fig:system_design_components}). Welche Daten erfasst werden und an welcher Stelle die Erhebung erfolgt, wird vollständig durch das Framework gesteuert.

Die Anwendung selbst muss im Wesentlichen lediglich die erforderlichen Daten über eine definierte Schnittstelle bereitstellen. Dadurch lässt sich das System flexibel in unterschiedlichen Technologien einsetzen, beispielsweise in WPF (siehe Unterabschnitt \ref{subsec:WPF}) oder Windows Forms (siehe Unterabschnitt \ref{subsec:Winforms}).

\subsection{Kommunikation zwischen Komponenten}

\subsubsection{Kommunikationsgrundlage}
Damit ein Informationsaustausch zwischen den Komponenten möglich ist, wird eine gemeinsame Kommunikationsbasis benötigt. Diese Grundlage besteht aus Objekten, die die gemeinsame Sprache für den Datenaustausch definieren.  
Konzeptionell wird zwischen fünf Informationskategorien unterschieden:

\begin{itemize}
    \item \textbf{Trackingaktionen}: Repräsentieren auftretende Ereignisse innerhalb der Anwendung.
    \item \textbf{Trackingdaten}: Umfassen Daten, die auf Basis einer aufgetretenen Aktion erfasst werden.
    \item \textbf{Extraktionsaufgaben}: Beschreiben, welche Informationen aus Trackingaktionen und Trackingdaten ermittelt werden sollen, und stellen Metadaten für die resultierenden Extraktionsdaten bereit.
    \item \textbf{Trackingaufgaben}: Legen fest, welche Daten für die Weiterverarbeitung ermittelt und verarbeitet werden dürfen.
    \item \textbf{Extraktionsdaten}: Enthalten die aus Trackingaktionen und Trackingdaten extrahierten Informationen, die anschließend weiterverarbeitet oder übermittelt werden können.
\end{itemize}

Die einzelnen Kategorien können verschiedene Objekttypen enthalten, die je nach Anwendungsfall unterschiedlich ausgestaltet sind. Diese Objekte bilden einen grundsätzlich unveränderlichen Standard, wobei Extraktionsdaten, Trackingaktionen und Trackingdaten durch benutzerdefinierte Objekte erweitert werden können. Damit die Daten vom Framework verarbeitet werden können, müssen sie, ähnlich wie bei Google Analytics (siehe Unterabschnitt \ref{subsec:google_analytics}), einer festgelegten Struktur entsprechen.

\subsubsection{Datenübertragungswege}
Der Datenaustausch zwischen den Komponenten kann entweder direkt über Rückgabewerte von Funktionen oder über sogenannte Datenkanäle erfolgen.  
Ein Datenkanal ist im Rahmen dieser Arbeit als ein Konstrukt zu verstehen, über das Daten an eine unbekannte oder externe Komponente weitergegeben werden können.  
Durch die Systemkonfiguration (siehe Abschnitt \ref{sec:integration_concept}) kann die Anwendung verschiedene Wege der Veröffentlichung oder des Bezugs von Daten anbieten. Auf diese Weise bleibt das System flexibel gegenüber unterschiedlichen Datenquellen und Zielen.

\subsubsection{Ablauf der Kommunikation}
Der Ablauf der Kommunikation erfolgt über die zuvor beschriebenen Datenobjekte und wird im in Abbildung~\ref{fig:sequence_diagram_communication_components} dargestellten Sequenzdiagramm veranschaulicht.  
Wie dort gezeigt, erzeugt die Anwendung zunächst alle benötigten Komponenten. Die Reihenfolge ergibt sich aus den jeweiligen Abhängigkeiten, wobei die genaue Erstellungsreihenfolge zwischen Datenermittlung (Abschnitt \ref{sec:data_collection_concept}), Verarbeitung (Abschnitt \ref{sec:data_extraction_concept}) und Trackingkonfiguration (Abschnitt \ref{sec:configuration_concept}) nicht entscheidend ist.  

Nach der Erzeugung wird der Tracking-Manager initialisiert. Während dieser Initialisierung wird auch die Systemkonfiguration (siehe Abschnitt \ref{sec:integration_concept}) übergeben. Anschließend initialisiert der Tracking-Manager die Trackingkonfiguration, die wiederum die benötigten Assemblies für die Konfiguration ermittelt. Danach kann der Tracking-Manager die Aufgaben für das Tracking anfordern. Diese Aufgaben werden bei der Initialisierung der Komponenten für Datenermittlung sowie für Verarbeitung und Extraktion berücksichtigt.

Nach Abschluss der Initialisierung kann die Anwendung sogenannte Data Provider registrieren. Dabei entscheidet die Komponente zur Datenermittlung, ob ein Provider einen entsprechenden Datenkanal zur Übertragung von Daten erhält. Über diesen Kanal können sowohl Aktivitätsdaten als auch nachgelieferte Daten gesendet werden.

Die übergebenen Daten werden anschließend in der Extraktionskomponente asynchron zur Anwendung empfangen und weiterverarbeitet.  

Wenn die Anwendung das Ausliefern der Daten anstößt, fordert der Tracking-Manager die Extraktionsdaten an und übergibt sie über den in der Systemkonfiguration bereitgestellten Datenkanal an die entsprechende Zielkomponente.

\begin{figure}[H]
    \centering
    \includegraphics[width=1\textwidth]{5_Sequence_Diagram_Components_Communication.pdf}
    \caption{Ablauf der Kommunikation zwischen den Komponenten}
    \label{fig:sequence_diagram_communication_components}
\end{figure}

\section{Tracking Konfiguration}
\label{sec:configuration_concept}

\section{Daten- und Aktionsermittlung}
\label{sec:data_collection_concept}

\section{Filterung und Extraktion}
\label{sec:data_extraction_concept}

\section{Systemkonfiguration}
\label{sec:integration_concept}


