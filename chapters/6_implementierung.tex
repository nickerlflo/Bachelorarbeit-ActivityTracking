\chapter{Implementierung}
\label{cha:implementierung}
In diesem Kapitel werden Teile der im Kapitel \ref{cha:konzept} beschriebenen Komponenten umgesetzt. Dabei werden exemplarisch Beispiele für ausgewählte Komponenten vorgestellt.  
Die vollständige Implementierung ist deutlich komplexer, die zugrunde liegenden Konzepte bleiben jedoch identisch, weshalb eine exemplarische Darstellung ausreichend ist.

\section{Konfiguration}
\label{sec:configuration_impl}
Im diesem Abschnitt wird beschrieben, wie die Konfiguration der zu trackenden Daten bereitgestellt und vom Manager der Konfiguration verarbeitet wird. Dabei werden die in Abschnitt \ref{sec:configuration_concept} besprochen Konzepte aufgearbeitet.

\subsection{Bereitstellung der Konfiguration}
Die Konfiguration wird pro Assembly (siehe Unterabschnitt \ref{subsec:assemblies}) bereitgestellt. Hierzu wird ein Provider im Projekt hinterlegt, der die Logik für die Ansicht und das zugehörige Presentationsmodell enthält.

Dieser Provider implementiert, wie in Programm \ref{prog:settings_provider} gezeigt, das Interface IActivityTrackingSettingsProvider mit zwei generischen Parametern. Diese Parameter definieren die Basistypen der Ansicht und des Presentationsmodells. Dies ist notwendig, um wie in Zeile 7 gezeigt zu überprüfen, ob es sich bei den angegebenen Typen tatsächlich um eine Ansicht handelt. Auf diese Weise wird der Nutzer vor einer falschen Verwendung geschützt.

Der StartSettingAccessor bildet den Einstieg in die Fluent API, über die Settings erstellt und angewendet werden können. Anschließend wird die \texttt{ProvideSettings}-Methode über Reflection (siehe Unterabschnitt \ref{subsec:reflection}) aufgerufen. Dadurch wird der Builder (siehe Unterabschnitt \ref{subsec:builder_implementation}) mit den entsprechenden Settings befüllt.

Im Beispiel aus Programm \ref{prog:settings_provider} wird sichergestellt, dass in der Aufgabenansicht erfasst wird, wie oft eine Aufgabe dupliziert wird. Hierbei werden sowohl die Bezeichnung der Funktionalität als auch ein Text für die spätere Datenanreicherung angegeben.

\begin{program}[H]
\begin{CsCode}
public class KISWinformsActivityTrackingSettingsProvider 
    : IActivityTrackingSettingsProvider<IKISStandalonePresenter, IView>
{
    public void ProvideSettings(
        IActivityTrackingStartSettingAccessor<IPresenter, IView> startSettingAccessor)
    {
        startSettingAccessor
            .TrackView<AufgabenView>()
            .GetUsage()
            .TrackFunctionality("Duplizieren", "Aufgabe dupliziert")
            .ApplySetting();
    }
}
\end{CsCode}
\caption{Bereitstellung der Tracking-Konfiguration.}
\label{prog:settings_provider}
\end{program}

\subsection{Emittelung der Konfiguration}


% Assemblies Referenzieren
% Assembly Resolver Refernzieren
% Reflection Referenzieren
% Nullreferenz Types Referenzieren

\subsection{Fluent API mit Builder}
\label{subsec:builder_implementation}
%Beispiel Settings
%Beispiel Builder
%Beispiel Fluent API in Anwendung erwähnen

\section{Daten- und Aktionsermittlung}
\label{sec:data_collection_impl}

\subsection{Aktion Provider Template}

\subsection{Auftragsabarbeitung}

\section{Filterung und Extraktion}
\label{sec:data_extraction_impl}

%Asynchrone Programmierung Referenzieren

\subsection{Verarbeitungskette Ablauf}

\subsection{Extraktion Beispiel}

\section{Integration WPF}
\label{sec:integration_wpf_impl}

\subsection{Schritt zur Integration}
%MVVM Refernzieren
%Erweiterungen Referenzieren

\subsection{Beispiel Aktionen Provider}
%Eventsystem Referenzieren

\section{Integration Windows Forms}
\label{sec:integration_winforms_impl}

\subsection{Schritt zur Integration}
%Aufgabendelegation Referenzieren

\subsection{Beispiel Daten Provider}
%Referenz auf Reflection

