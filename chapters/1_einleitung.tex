\chapter{Einleitung}
\label{cha:introduction}
Dieses Kapitel gibt eine Übersicht über die Motivation und die Ziele dieser Bachelorarbeit. 
Des Weiteren wird ein Überblick über das Vorgehen zur Erreichung dieser Ziele gegeben.

\section{Hintergrund und Motivation}

\subsection{Umstellung von Softwaresystemen}
Viele Unternehmen stehen vor der Entscheidung, größere Softwaresysteme auf neue Technologien umzustellen. Dies erfordert jedoch häufig einen enormen Entwicklungsaufwand und somit oft mehr Aufwand, als die verfügbaren Ressourcen zulassen, da neben der Umstellung meist noch neue Produkte entwickelt werden und alte Software gewartet wird. Neben KI, speziellen Tools und anderen technischen Möglichkeiten, Code wiederzuverwenden oder auf eine andere Technologie zu migrieren, ist eine vorerst einfache Möglichkeit, Programmfunktionalität wegzulassen.

\subsection{Entscheidung für Programmfunktionalität}
Die Basis für faktenbasierte Entscheidungen bilden hierbei Verhaltensdaten aus der bestehenden Anwendung. Derartige Daten müssen daher vor solchen Umstellungen mit Tracking-Technologien ermittelt und ausgewertet werden. Im Web-Umfeld gibt es bereits verschiedene Möglichkeiten (siehe Abschnitt \ref{sec:similar_applications}), gezielt solche Daten zu sammeln und auf einer Weboberfläche darzustellen. Vergleichbare Möglichkeiten existieren im Bereich von Desktopanwendungen, die auf den UI-Frameworks WPF und Windows Forms basieren, nicht. Die Auswertung mit den bestehenden Systemen ist zwar oft unabhängig von der Technologie möglich, jedoch fehlt das Gegenstück zur Sammlung der Daten aus Desktopanwendungen.

\subsection{Technologieänderung Richtung Web}
Aktuelle Trends zeigen, dass sich die Softwareentwicklung zunehmend in Richtung Web verlagert, wodurch klassische Desktopanwendungen an Relevanz verlieren. Besonders deutlich wird dies am starken Wachstum im Bereich Software as a Service (SaaS) und Cloud-Lösungen \cite{konsgen2018key}, die in den letzten Jahren erheblich an Bedeutung gewonnen haben. Vor diesem Hintergrund ist es gerade für nicht webbasierte Technologien wichtig, die Nutzung der Software systematisch zu erfassen. Unternehmen wie die RZL Software GmbH benötigen hierfür Lösungen, die es ermöglichen, Nutzungsdaten in WPF- und Windows-Forms-Applikationen möglichst einfach und flexibel zu sammeln, ohne tiefgreifende Eingriffe in bestehende Systeme vornehmen zu müssen.

\section{Ziele der Arbeit}

\subsection{Datenermittlung}
Das Ziel der Arbeit ist die Entwicklung eines Verhaltens-Tracking-Frameworks, das für Windows-Forms- und WPF-Anwendungen eingesetzt werden kann. Die Anforderungen an dieses System sind die Ermittlung, Filterung und Extraktion von Nutzungsdaten. Darauf aufbauend müssen die ermittelten Daten strukturiert und aggregiert an ein bestehendes System übermittelt werden können. Daher liegt der Fokus der Arbeit auf der Datenermittlung und nicht der Auswertung oder Darstellung der Daten.

\subsection{Konfigurierbarkeit}
Aus Entwicklungssicht muss eine gezielte Steuerung und Auswahl der zu sendenden Daten einfach und ohne Anpassung des bestehenden Codes möglich sein. Somit entsteht die Voraussetzung einer einmaligen Integration in das bestehende System, wobei sich das Framework flexibel durch Konfiguration an das Basissystem anpassen muss.

\subsection{Fragen zur Beantwortung}
\label{subsec:initial_questions}
Von besonderer Relevanz ist, dass die vom System erfassten Informationen die Auszuwertenden dabei unterstützen, folgende Fragen zu beantworten:

\begin{itemize}
    \item Wie oft wird eine Ansicht in der Anwendung geöffnet?
    \item Wie oft wird eine bestimmte Anwendungsmöglichkeit verwendet?
    \item Welche Shortcuts werden in einer Ansicht verwendet?
    \item Wie hoch ist die Anzahl von bestimmten Datensätzen?
    \item Wie beeinflussen Versionsänderungen das Laufzeitverhalten?
    \item Welche Aktionen treten in einer Ansicht auf?
    \item Gibt es eine bestimmte Abfolge von Aktionen im System?
    \item Wie interagieren Nutzer mit einer Ansicht?
\end{itemize}

\subsection{Anwendung des Systems}
Um zu zeigen, dass dieses System die entsprechenden Anforderungen erfüllt, soll diese Arbeit diese Fragen aus Abschnitt \ref{subsec:initial_questions} auf Basis einer Ansicht aus einem bestehenden System exemplarisch beantworten. Somit soll an einem Beispiel die Verwendung des Frameworks verdeutlicht werden.

\section{Vorgehensweise}
Um ein funktionierendes Framework zu entwickeln, welches die Anforderungen erfüllt, wird nach einem Schema vorgegangen. Dieses Schema ist in diesem Abschnitt beschrieben.

\subsection{Analyse}
\label{subsec:research}
Im ersten Schritt ist es wesentlich, bestehende Konzepte aus Systemen zu erfassen und auf die verwendete Technologie zu übertragen. Diese Systeme werden größtenteils im Webbereich eingesetzt und sind in Abschnitt \ref{sec:solutions_tracking} zusammengefasst. Aus diesen Grundlagen geht auch hervor, wie wichtig es ist, Cross-Cutting-Concerns \cite{crosscutting} zu berücksichtigen und den bestehenden Code dabei qualitativ hochwertig zu halten.

\subsection{Konzept}
\label{subsec:design_info}
Im nächsten Schritt ist es notwendig, die im Abschnitt \ref{subsec:research} gewonnenen Erkenntnisse in ein möglichst modulares System zu überführen. Dabei wird mit der Konfiguration der Datensammlung begonnen, da sie klar beschreibt, welche Aufgaben die anderen Komponenten im System erfüllen müssen. Die Komponenten für die Datenerfassung, das Extrahieren, die Zusammenstellung der Daten sowie die Konfiguration des Frameworks im Kontext der Anwendung folgen anschließend.

\subsection{Implementierung}
Das aus Abschnitt \ref{subsec:design_info} erhaltene Konzept kann im Anschluss umgesetzt werden. Dabei ergibt sich die zum Testen und Entwickeln logische Reihenfolge: Konfiguration, Datensammlung, Datenextraktion und Integration in Anwendungen. Bei der Umsetzung wird auf die wichtigsten Bestandteile des Frameworks und dessen Implementierung eingegangen, wobei an Ausschnitten aus dem Framework die Umsetzung erläutert wird.

\subsection{Konfiguration und Testen}
Um die Funktionsfähigkeit des Systems zu demonstrieren, wird eine Konfiguration bereitgestellt. Die daraus resultierenden Daten werden anschließend verwendet, um die in Abschnitt \ref{subsec:initial_questions} abgeleiteten Fragen exemplarisch zu beantworten.


