\chapter{Kurzfassung}
\label{cha:abstract}

\begin{german}

Der schnelle technologische Wandel erfordert von Unternehmen wie der RZL Software GmbH eine kontinuierliche Anpassung und stellt sie vor strategischen sowie technischen Herausforderungen. Die Umstellung älterer Softwaresysteme bedeutet in der Regel einen enormen Ressourcenaufwand. Hinsichtlich der Effizienz kann eine gezielte Auswahl von Funktionalitäten auf Basis von Nutzerdaten wesentliche Steigerungen bewirken. Des Weiteren können etablierte Prozesse bei Anwendern des bestehenden Softwareprodukts ermittelt, beibehalten und optimiert werden.

Während es im Web-Umfeld bereits bestehende Standardlösungen zum Ermitteln von Nutzerdaten gibt, ist dies für komplexe Windows-Forms- und WPF-Rich-Client-Applikationen nicht der Fall. Daher beschäftigt sich diese Bachelorarbeit mit der Entwicklung eines Aktivitäts-Tracking-Frameworks, das für WPF- und Windows-Forms-Anwendungen flexibel eingesetzt werden kann. Dieses Framework soll es ermöglichen, Verhaltensdaten an das Unternehmen zu senden, ohne gravierende Änderungen am bestehenden System vorzunehmen und dabei die Benutzung des Programms nicht zu beeinträchtigen.

Die Arbeit beginnt mit dem Vergleich verschiedener Technologien und dem Erarbeiten der Grundlagen. Darauf aufbauend wird das Konzept für das Framework erstellt. Anschließend wird das System mittels C\# umgesetzt und zu Testzwecken jeweils in eine WPF- und eine Windows-Forms-Applikation integriert. Um die Funktionsfähigkeit zu demonstrieren, wird eine Konfiguration erstellt und aus dem bestehenden System entsprechende Daten ermittelt. Diese Daten bilden zugleich die Basis für die exemplarische Beantwortung von Fragen, die sich aufgrund des Optimierungsproblems ergeben haben.

\end{german}