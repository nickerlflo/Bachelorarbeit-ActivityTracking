\chapter{Technologien und Werkzeuge}
\label{cha:technologie_werkzeuge}
Die verwendeten Technologien und Werkzeuge sind für diese Arbeit wesentlich, um die Implementierung nachvollziehen zu können. Die Entwicklungsplattform .NET spielt dabei für das entwickelte Framework eine entscheidende Rolle, wie bereits in Unterabschnitt \ref{subsec:patterns} bei den beschriebenen Patterns erwähnt. In diesem Kapitel werden daher einige zentrale Konzepte vorgestellt. Der Schwerpunkt liegt insbesondere auf der Behandlung der verwendeten UI-Frameworks der .NET-Plattform.

\section{Visual Studio}
\label{sec:visual_studio}

\section{.NET}
\label{sec:dotnet}
Für die Umsetzung der Implementierung in Kapitel \ref{cha:implementierung} wird die von Microsoft bereitgestellte Laufzeitumgebung und Entwicklungsplattform .NET \cite{dotnet} in der Version 8 \footnote{Microsoft veröffentlicht alle zwei Jahre eine Long-Term-Support-Version (LTS) von .NET. Version 8 wurde im November 2024 bereitgestellt und bietet langfristigen Support. Darüber hinaus erscheint jährlich im November eine neue .NET-Version.} verwendet. Diese Version enthält bereits die beiden UI-Frameworks WPF (siehe Unterabschnitt \ref{subsec:WPF}) und Windows Forms (siehe Unterabschnitt \ref{subsec:Winforms}), die ausschließlich unter Windows-Betriebssystemen eingesetzt werden können. Die für diese Arbeit relevanten Konzepte und Funktionen von .NET werden in diesem Kapitel näher erläutert.

\subsection{C\#}
\label{subsec:csharp}
Für die .NET-Plattform stehen mehrere Programmiersprachen zur Verfügung, darunter C\#, F\# und VB.NET. Für die Entwicklung des Frameworks wird C\# verwendet. Diese Sprache gilt laut Microsoft \cite{microsoft-tour-of-csharp} als die beliebteste Sprache innerhalb des .NET-Ökosystems und basiert auf dem objektorientierten Paradigma. Darüber hinaus integriert C\# Konzepte aus weiteren Programmierparadigmen, wie der funktionalen Programmierung. Ein Beispiel für die funktionale Programmierung ist Language Integrated Query (LINQ), das typisierte Abfragen in einer SQL-ähnlichen Syntax direkt im Code ermöglicht.

\subsubsection{Übersetzung und Ausführung}
C\# wird, wie auch andere Sprachen desselben Ökosystems, zunächst in IL (Intermediate Language) übersetzt. Dieser IL-Code wird anschließend durch die Common Language Runtime (CLR) \cite{microsoft-clr-overview} in nativen Maschinencode umgewandelt. Standardmäßig erfolgt diese Übersetzung in .NET über den Just-in-Time-Compiler (JIT). Es existieren jedoch auch Varianten, bei denen Anwendungen bereits teilweise oder vollständig vor der Ausführung in nativen Code kompiliert werden (Ahead-of-Time-Kompilierung \cite{microsoft-native-aot}).

Für die Ausführung von .NET-Programmen wird somit die CLR benötigt, die Bestandteil der .NET Runtime Environment bzw. des .NET SDK ist. Die Assemblies (siehe Unterabschnitt \ref{subsec:assemblies}) werden üblicherweise als \texttt{.dll}- oder \texttt{.exe}-Dateien abgelegt, unterscheiden sich inhaltlich jedoch von C++ Programmen, die nativ kompiliert werden.

\subsection{Nullable Reference Types}
Der bekannte Informatiker Tony Hoare bezeichnete die Einführung von \texttt{null} als seinen größten Fehler, da die dadurch verursachten Probleme enorm seien \cite{hoare_null_video}. Dieses Problem lässt sich bis zu einem gewissen Grad durch die sogenannten Nullable Reference Types \cite{microsoft-nullable-references} lösen. Der Compiler prüft hierbei, ob eine Variable den Wert \texttt{null} annehmen kann und falls dies der Fall ist, muss der Code explizit eine entsprechende Überprüfung vorsehen. Aufgrund der damit verbundenen Reduzierung von NullReferenceException-Fehlern wird dieses Konzept auch in dieser Arbeit verwendet. Entscheidend ist jedoch, dass die durch den Compiler entstehenden Warnings auf Errors umgestellt werden, um die Abdeckung aller entstandenen Nullszenarien zu garantieren.

\subsection{Assemblies}
\label{subsec:assemblies}

\subsection{Asynchrone Programmierung}
\label{subsec:async}

\subsection{Reflection}
\label{subsec:reflection}

\section{UI Frameworks}
\label{sec:ui_frameworks}

\subsection{WPF}
\label{subsec:WPF}

\subsection{Windows Forms}
\label{subsec:Winforms}


